\chapter{Introduction}

The current set of data on plant identification is geared towards direct research of machine capabilities or large scale food production and distribution.
Plant identification through computer vision technologies has been primarily  researched in controlled settings with specific  views of the plants. This approach has value and merits but is still limited in both scope and application as most models focus on a limited number of plants or provide only a top down perspective of each plant.

Home gardeners, market gardens, and small sized farms produce a wide variety of plants using a multitude of methods. Gardens are propagated in various mediums such as raised beds, trellises, and tabled trays or towers. These plants are inspected by the caretaker at parallel and oblique angles  

The goal is to build a data-set of photos and labels of garden variety plants viewed at an oblique angle, for identification in an outdoor environment. Using Machine Learning algorithms we will be testing for identification accuracy against varied plant types, for example, differentiating between a carrot plant and a beet plant, and between plant breeds, for example, differentiating  a Roma tomato plant from a Beefeater tomato plant before fruiting. First we will use a binary classification of CropPlant/Invasive Plant. Determining a crop plant will allow for further categorization through labeling the plant type. A final classification will attempt to differentiate the plant type breed.

First we will use a binary classification of CropPlant/InvasivePlant. Determining a crop plant will allow for further categorization through labeling the plant type. A final classification will attempt to differentiate the plant types breed.

\begin{enumerate}
\item Binary Classification - Invasive Plant / Crop Plant
\item Crop identification - Tomato, Beet, Carrot, Watermelon
\item Breed identification - Roma, Beefeater, Cherry 
\end{enumerate}

\section{Problem Statement} \label{sec:problem-statement}
We describe the XXX problem as follows: ...

 All problem statements include the following:
 \begin{description}
     \item[GIVEN:] What is the \emph{input} or what are the \emph{initial conditions}
   \item[SUCH THAT:] What is the scope? What are the limitations on the input that
                you need to impose in order to be able to solve it?
   \item[AIM:] What is your \emph{goal}?   What do you want to accomplish?
   \item[CRITERA:]  How can you measure whether you have done a good job? This needs to
               be something that you can measure. It is possible to use even
               subjective things like "is this fake image realistic" by using
               a user study -- in that case criteria would be that
               a human user cannot distinguish real from fake images X\% of the time.
 \end{description}
 It is up to you whether you call out those parts or just use POE (plain old English)

  You will \emph{probably} want to refine your problem statement to include several other
  sub-problems that need to be solved.  These can form the basis of multiple contributions
  or chapters later in the work.

\section{Contributions} \label{sec:contributions}
The thesis of this work is ....
 re-word contribution \ref{itm:first}, it will be your thesis.

In particular, we
expect   %<-- comment this line out for the final thesis.
% make   %<-- uncomment this for the final Thesis (after your proposal)
the following contributions:
\begin{enumerate}
    \item \label{itm:first} List your main, most important contribution here. This is the "real" one
    \item  List a secondary contribution. This one is likely less signficant than the
           first one, but DO NOT admit that to ANYONE
    \item  And list your weakest one here. Always good to have three -- try to make sure
           you have 3 plausible things -- but the first one should be the winner
\end{enumerate}


% The last paragraph explaines the stryucture of the rest of the Thisis.
The rest of the thesis is organized as follows....



